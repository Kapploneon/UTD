\documentclass[varwidth]{standalone}

\begin{document}


\section{Longest Palindromic Substring}
\subsection{Brute Force}
First idea is just brute force.

\begin{quote}
    Traverse the string.

    For each character, find the longest palindromic substring.

    In order to do that, traverse the substring began at that character.
    If a certain substring is palindromic, mark the length and continue.
\end{quote}
The biggest problem for this brute force solution is two characters in
the string might be repeatedly compared in all n round.
In all the effort to optimize this solution, it still exceeds time limits
occasionally.

\subsection{Expanding}
So here comes the second intuitive idea.

\begin{quote}
    Consider a character at index $i$ in the middle of the string.

    If it is the center of a palindromic substring,
    then all the characters at its left and right must be mirrored.
    Therefore, we can check the characters at $i\pm k$ recursively.

    If we traverse the string, perform the previous operations,
    we can calculate all the palindromic substrings centered at each character.
\end{quote}
This idea hugely reduce comparisons. Since each pair of characters can only
be centered at one character. And each character is visited only once.
Thus each pair of characters are compared only once.
However, it contains an obvious ambiguity: What if a palindromic is even in
length? There is no center character for an even length string.

To generalize the solution, here comes a modification:
Use a series of same characters as center, rather than an aimless character.
Here is the optimized solution.
\begin{enumerate}
    \item Traverse the string, continue if same character found, stop until finding
    a different character.

    \item View the series of same characters just found as center, expanding at both
    direction, until find unmatching character. The substring between two ends
    is a palindromic substring. Mark length and continue at the different
    character found in 1.
\end{enumerate}
This solution turns out to be even more efficient than the original idea since long
series of same character can be stepped over.

\subsection{Longest Common Substring}
Although the previous solution is quite efficient, we can still view the
problem from another aspect.
\begin{quote}
    If we reverse the original string, the palindromic substring must be
    a common string of the reversed and original string.
    In this way, finding the longest palindromic substring is equals to
    finding the longest common substring between two string.
\end{quote}
Nevertheless, the idea is not correct in all circumstances. As an example given
in the editorial:
\begin{itemize}
    \item $S = \mathtt{''abacdfgdcaba''}$
    \item $S_r = \mathtt{''abacdgfdcaba''}$
\end{itemize}
Obviously, the common substring ``abacd'' is not palindromic.
A quick fix to this problem is check every time we find a common substring.

To solve the longest common substring problem, we can implement a
$\mathcal{O}(n^2)$ dynamic programming solution, using $\mathcal{O}(n^2)$ or
$\mathcal{O}(n)$ space.

\subsection{To-Do}
Besides the solution above, an even more efficient solution named 
Manacher's Algorithm is available, which is left as to do in the section.


\end{document}

