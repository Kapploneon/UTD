\documentclass[varwidth]{standalone}

\begin{document}

\section{Combination (P077)}

Intuitive idea is first generate combination including current integer, by
recursively call combine with $s+1$ and $k-1$. And recursively generate
combination excluding current integer. Finally, return the union of two list.
Algorithm is shown in \cref{077:rec}.

\begin{algorithm}[H]
    \caption{Intuitive Recursive Solution}\label{077:rec}
    \begin{algorithmic}[1]
        \Procedure{Combination}{$s,n,k$}
            \If{$k = 0$}
                \State\Return {List of Empty List}
            \EndIf
            \State $include \leftarrow \ProcedureName{Combination}{s+1,n,k-1}$
            \State \ProcedureName{Map}{(l \rightarrow l.add(s)), include}
            \State $exclude \leftarrow \ProcedureName{Combination}{s+1,n,k}$
            \State\Return{$include \cap exclude$}
        \EndProcedure
    \end{algorithmic}
\end{algorithm}

Note that, this solution is designed to be transformed to DP. Each recursive
call is strictly rely on a larger $s$ and a less or equal $k$. By memoizing the
result of each recursive call, we can reduce time complexity with cost of some
increase in space complexity. However, during implementation, I realized that
Java do not support generic type array. Variable
\texttt{List<List<Integer>>[][] m} can never be initialized.

Another thought is backtracking. Just like DFS, each time, a recursive call is
aware its current status, if reaching the leaf node, the program add current
path from root to leaf to the final return list, and then backtrack to upper
level and continue traversing.

This solution is relatively faster because there are less recursive call.
However, it requires the final list be a global variable (or passed as
parameter by reference in the recursive call), so does the variable storing 
current state, which I believe is not a very good approach.


\end{document}

