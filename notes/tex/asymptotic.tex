
\section{Asymptotic Notation -- big ``O'' notation}

\subsection{Growth of Functions}

The growth of function in Table~\ref{function_list} increase downwards.
\begin{table}[H]
\centering
\caption{Function List}\label{function_list}
\begin{tabular}{c|l}
$\log_{10} n$ & binary search \\
$n$ & input \\
$n^2$ & pairs \\
$10^{10}n^{10}$ & \\
$1.000.1^n$ & \\
$2^n$ & Binary string of length n \\
$n!$ & Permutation \\
\end{tabular}
\end{table}
Let $f(n)$, $g(n)$ be function.

\subsection{big ``O'' notation}
\begin{definition}

$f(n) = \mathcal{O}(g(n))$, if $\exists n_0 \in \mathbb{N}$,
$c \in \mathbb{R}^+$, s.t. $\forall n \geq n_0$, $f(n) \leq c * g(n)$,
and $\lim_{n \rightarrow \infty} \frac{f(n)}{g(n)} \ne \infty$, i.e.\ it is $\lim_{n \rightarrow \infty} \frac{f(n)}{g(n)} < k$, for some constant $k$.
\end{definition}

Table \ref{def_asymptotic_notation} shows the basic definition of all the asymptotic notations.
\begin{table}[H]
\centering
\caption{Definition for all Asymptotic Notation}\label{def_asymptotic_notation}
\begin{tabular}{c|l|c}

\hline
$f(n)$ & $\lim_{n \rightarrow \infty} \frac{f(n)}{g(n)}$ & relation \\
\hline
\hline
$\mathcal{O}(g(n))$ & $\neq \infty$ & $\leq$ \\
$\Omega(g(n))$  & $\neq \infty$ & $\leq$  \\
$\Theta(g(n))$  & $= k > 0$ & $=$  \\
$o(g(n))$  & $= 0$ & $<$  \\
$\omega(g(n))$  & $= \infty$ & $>$  \\
\end{tabular}
\end{table}

\subsection{Asymptotic Relation's feature}
\begin{theorem}
Multiplying by positive constant does NOT change asymptotic relations.
i.e.\ if $f(n) = \mathcal{O}(g(n))$, then $100 * f(n) = \mathcal{O}(g(n))$.
\end{theorem}

\begin{proof}
$f(n) = \mathcal{O}(g(n)) \Rightarrow \exists{n_0} \exists c, \forall n \geq n_0, f(n) \leq c*g(n)$,

then, $\exists{n_0}\exists{c'}$, s.t.\ $\forall n \geq n_0$, $100*f(n) \leq c'*g(n) = 100c*g(n)$.
\end{proof}

Example:
\begin{align}
C*2^n & = \Theta(2^n) \\
(C*2)^n & \neq \Theta(2^n)
\end{align}

\begin{claim}
Show: $2n\log(n) - 10n = \Theta(n\log(n))$
\end{claim}

\begin{proof}
First show: $2n\log(n) - 10n = \mathcal{O}(n\log(n))$

For $n_0 = 1$, $c = 2$

\[2n\log(n) - 10n \leq 2n\log(n)\]

Now show: $2n\log(n) - 10n = \Omega(n\log(n))$

For $n_0 = 2^10$, $c = 1$,

\[
\begin{split}
2n\log(n) - 10n & \geq n\log(n) + n \log(2^{10}) - 10n \\
& =n \log(n) + 10n - 10n \\
& =n\log(n) \\
\end{split}
\]

$n_0 = 1$ ($n_0 = 2^{10}$) means $n$ is at least $1$ (or $2^10$).
\end{proof}

\begin{corollary}
$\mathcal{O}(1)$ means \textbf{Any Constant}.
\end{corollary}

\textbf{Attention}: \emph{Asymptotic notation has limit. It is not applicable for all scenarios}.

\subsection{Properties of $\log(n)$}
\definition{$n = C^{\log_c{n}}$, $c > 1$, $\lg{n} = \log_2{n}$, $\ln{n} = \log_e{n}$.}

\begin{theorem}
$\forall{a,b}  > 1$
\[
\begin{split}
\log_b(n) & = \frac{\log_a(n)}{\log_a(b)} \\
\log_b(n) & = \Theta({\log_a(n)}) \\
\end{split}
\]
\end{theorem}

\begin{theorem}
$\forall{a,b} \in \mathbb{R}$
\[\begin{split}
\log(a^n) & = n*\log(a)
\\
\log(a*b) & = \log(a) + \log(b))
\\
a^{\log(b)} & = b^{\log(a)}
\\
\end{split}\]
\end{theorem}

\theorem{$\lg(n)$ is to $n$ as n is to $2^n$.}

\subsection{Something More}

\theorem{Let $f(n)$ be a polynomial function, then $\log(f(n)) = \Theta(\log(n))$.}
\proof{The asymptotic result of $n^2$ and $n^10$ are the same.}

\begin{definition}
$\log^*(n) = o(\log\log\log\log\log(n)) = \alpha$.
\end{definition}


Example: $\lg^*(2^{2^{2^{2^{2}}}}) = 5$.
