\documentclass[12pt,letterpaper,titlepage,en-US]{article}

\usepackage{basicstyle}
\usepackage{homework}

%
% Homework Details
%   - Title
%   - Due date
%   - Class
%   - Section/Time
%   - Instructor
%   - Author
%


\newcommand{\hmwkTitle}{Homework\ \#4 }
\DTMsavetimestamp{DueDate}{2016-11-06T23:59:59-06:00}
\newcommand{\hmwkClass}{CS 6360.003}
\newcommand{\hmwkClassName}{Database Design}
\newcommand{\hmwkClassInstructor}{Nurcan Yuruk}
\newcommand{\hmwkAuthorName}{Hanlin He}
\newcommand{\hmwkAuthorNetID}{hxh160630}
\newcommand{\hmwkAuthorUTDEmail}{\href{mailto:hanlin.he@utdallas.edu}{hanlin.he@utdallas.edu}}


%
% Title Page
%

\title{
    \vspace{2in}
    \textmd{\textbf{\hmwkClassName \\\hmwkClass:\ \hmwkTitle}}\\
    \normalsize\vspace{0.1in}\small{Due\ on\ \DTMusedate{DueDate}\ at \DTMusetime{DueDate} }\\
    \vspace{0.1in}\large{\textit{\hmwkClassInstructor}}
    \vspace{3in}
}

\author{\textbf{\hmwkAuthorName\ \footnotesize{(\hmwkAuthorNetID)}} \\ \hmwkAuthorUTDEmail}
\date{}

\begin{document}
\maketitle


\pagebreak

\begin{homeworkProblem}
Let
\begin{align*}
    F_1 &= \{A \rightarrow C, AC \rightarrow D, E \rightarrow AD, EC \rightarrow DH, DE \rightarrow CH\} \\
    F_2 &= \{A \rightarrow CD, E \rightarrow AH\}
\end{align*}

For each attributes set on the left side of $F_1$, calculates its closure with respect to $F_2$:
\begin{align*}
    \{A\}^+ &= \text{A, C, D} \quad \text{including $C$ in $F_1$}\\
    \{AC\}^+ &= \text{A, C, D} \quad \text{including $D$ in $F_1$}\\
    \{E\}^+ &= \text{A, C, D, H} \quad \text{including $A, D$ in $F_1$}\\
    \{EC\}^+ &= \text{A, C, D, E, H} \quad \text{including $D, H$ in $F_1$}\\
    \{DE\}^+ &= \text{A, C, D, E, H} \quad \text{including $C, H$ in $F_1$}\\
\end{align*}

Thus, $F_2$ covers $F_1$.

Respectively,consider the attributes sets on the left side of $F_2$:
\begin{align*}
    \{A\}^+ &= \text{A, C, D} \quad \text{including $C, D$ in $F_2$} \\
    \{E\}^+ &= \text{A, C, D, E, H} \quad \text{including $A, H$ in $F_2$} \\
\end{align*}

Thus, $F_1$ covers $F_2$.

Hence, $F_1$ is equivalent to $F_2$.

\end{homeworkProblem}

\begin{homeworkProblem}
    Based on Algorithm 16.2(a), it is easy to see that
    \begin{align*}
        \{AD\}^+ &= \{A,B,C,D,E,F,G,H,I,J\} \\
        \{DG\}^+ &= \{A,B,C,D,E,F,G,H,I,J\}
    \end{align*}

    So $\{AD\}$ and $\{DG\}$ are candidate keys of relation $R$, $\{A, D, G\}$ is the prime attribute. Let $\{AD\}$ be the primary key.
    \begin{align*}
        \{A\}^+ &= \{A, I\}\\
        \{D\}^+ &= \{D\}
    \end{align*}

    Apparently, only one attribute is partially functional dependent on keys,
    which is $I$ with functional dependency $A \rightarrow I$.

    So the following relation would be in 2NF:
    \begin{align*}
        R_1& : \{\underline{A}, B, C, \underline{D}, E, F, G, H, J\}\\
        R_2& : \{\underline{A}, I\}
    \end{align*}

    $R_2$ is obviously in 3NF,
    consider the functional dependencies in $R_1$.
    \begin{itemize}
        \item $FD_1: \{AB \rightarrow C\}$, $AB$ is not a superkey of $R_1$,
            and $C$ is not a prime attribute,
            thus violates the general definition of 3NF.
        \item $FD_2: \{BD \rightarrow EF\}$, $BD$ is not a superkey of $R_1$,
            and $EF$ is not prime attributes,
            thus violates the general definition of 3NF.
        \item $FD_3: \{AD \rightarrow GH\}$, $AD$ is a superkey of $R_1$,
            thus does not violate the general definition of 3NF.
        \item $FD_4: \{GD \rightarrow ABH\}$, $GD$ is a superkey of $R_1$,
            thus does not violate the general definition of 3NF.
        \item $FD_5: \{H \rightarrow J\}$, $H$ is a non-prime attribute,
            and $J$ is another non-prime attribute,
            thus violates the general definition of 3NF.
    \end{itemize}

    So we could decompose $R_1$ based on $FD_1$, $FD_2$ and $FD_5$ to achieve 3NF.

    The final relations in 3NF are:
    \begin{align*}
        R_1& : \{\underline{A}, B, \underline{D}, G, H\}\\
        R_2& : \{\underline{A}, I\}\\
        R_3& : \{\underline{A}, \underline{B}, C\}\\
        R_4& : \{\underline{B}, \underline{D}, E, F\}\\
        R_5& : \{\underline{H}, J\}\\
    \end{align*}

\end{homeworkProblem}

\begin{homeworkProblem}
    \begin{enumerate}[label=\textbf{Step {\arabic*}}, leftmargin=2cm]
        \item Consider the right side of the functional dependencies.
            The original functional dependencies turns into:
            \begin{itemize}
                \item $AB \rightarrow C$
                \item $AB \rightarrow D$
                \item $AB \rightarrow E$
                \item $C \rightarrow B$
                \item $C \rightarrow D$
                \item $CD \rightarrow E$
                \item $DE \rightarrow B$
            \end{itemize}
        \item Consider the left side of the functional dependencies.
            \begin{itemize}
                \item $CD \rightarrow E$ can be replaced with $C \rightarrow E$ because $C \rightarrow D$.
            \end{itemize}
        \item Consider the redundancy of the functional dependencies.
            \begin{itemize}
                \item $AB \rightarrow C$ and $C \rightarrow D$, so $AB \rightarrow D$ is redundant.
                \item $AB \rightarrow E$ and $C \rightarrow E$, so $AB \rightarrow E$ is redundant.
                \item $C \rightarrow D$, $C \rightarrow E$ and $DE \rightarrow B$, so $C \rightarrow B$ is redundant.
            \end{itemize}
        \item Hence, the minimal cover would be:
            $\displaystyle\left\{
                    \begin{array}{rcl}
                        AB &\rightarrow& C \\
                        C &\rightarrow& DE \\
                        DE &\rightarrow& B
                    \end{array}
            \right\}$
    \end{enumerate}
\end{homeworkProblem}

\begin{homeworkProblem}
    \begin{homeworkSubProblem}
        \begin{enumerate}[label=\textbf{Step {\arabic*}}, leftmargin=2cm]
            \item Set $K = R = \{A,B,C,D,E,F,G,H,I,J\}$.
            \item Remove $A$ from $K$, $K^+ = \{A,B,C,D,E,F,G,H,I,J\}$, since $F \rightarrow A$.\\
                $K = \{B,C,D,E,F,G,H,I,J\}$ is still a superkey.
            \item Remove $B$ from $K$, $K^+ = \{A,B,C,D,E,F,G,H,I,J\}$, since $H \rightarrow B$.\\
                $K = \{C,D,E,F,G,H,I,J\}$ is still a superkey.
            \item Remove $C$ from $K$, $K^+ = \{A,B,C,D,E,F,G,H,I,J\}$, since $FI \rightarrow C$.\\
                $K = \{D,E,F,G,H,I,J\}$ is still a superkey.
            \item Remove $D$ from $K$, $K^+ = \{A,B,C,D,E,F,G,H,I,J\}$, since $HI \rightarrow D$.\\
                $K = \{E,F,G,H,I,J\}$ is still a superkey.
            \item Remove $E$ from $K$, $K^+ = \{A,B,C,D,E,F,G,H,I,J\}$, since $FI \rightarrow E$.\\
                $K = \{F,G,H,I,J\}$ is still a superkey.
            \item Remove $F$ from $K$, $K^+ = \{A,B,C,D,E,F,G,H,I,J\}$, since $HI \rightarrow F$.\\
                $K = \{G,H,I,J\}$ is still a superkey.
            \item Remove $G$ from $K$, $K^+ = \{A,B,C,D,E,F,G,H,I,J\}$, since $HI \rightarrow G$.\\
                $K = \{H,I,J\}$ is still a superkey.
            \item Remove $H$ from $K$, $K^+ = \{I,J\}$. \\
                $K = \{I,J\}$ is no longer a superkey, so $H$ cannot be removed.
            \item Remove $I$ from $K$, $K^+ = \{B,G,H,J\}$. \\
                $K = \{H,J\}$ is no longer a superkey, so $J$ cannot be removed.
            \item Remove $J$ from $K$, $K^+ = \{A,B,C,D,E,F,G,H,I,J\}$, since $HI \rightarrow FI \rightarrow J$.\\
                $K = \{H,I\}$ is a superkey, and cannot be divided any more.\\
                Thus $\{H, I\}$ is a candidate key.
            \item Repeat above operations in different deletion order. It is easy to get that:$\{F,I\}^+ = \{A,B,C,D,E,F,G,H,I,J\}$
                $K = \{F,I\}$ is a superkey, and cannot be divided any more.\\
                Thus $\{F, I\}$ is also a candidate key.
        \end{enumerate}
        In conclusion, $\{F, H, I\}$ are the prime attributes of relation $R$.

    \end{homeworkSubProblem}

    \begin{homeworkSubProblem}
        Based on the general definition of 3NF:
        \begin{itemize}
            \item $FD_1: FI \rightarrow EHJC$ forms a 3NF relation since $FI$ is a superkey of relation$R$.
            \item $FD_2: H \rightarrow GB$ violates the 3NF general definition since neither is $H$ a superkey nor $GB$ prime attributes for relation $R$.
            \item $FD_3: F \rightarrow EA$ violates the 3NF general definition since neither is $F$ a superkey nor $EA$ prime attributes for relation $R$.
            \item $FD_4: HI \rightarrow FGD$ forms a 3NF relation since $HI$ is a superkey of relation$R$.
            \item $FD_5: A \rightarrow C$ violates the 3NF general definition since neither is $A$ a superkey nor $C$ prime attributes for relation $R$.
        \end{itemize}

        Thus, we can decompose the relation $R$ directly into the following based on $FD_2$, $FD_3$ and $FD_5$:
        \begin{align*}
            R_1& : \{D, \underline{F}, H, \underline{I}, J\} & \text{Remaining attribute in R}\\
            R_2& : \{B, G, \underline{H}\} & \text{Decomposition based on $FD_2: H \rightarrow GB$}\\
            R_3& : \{A, E, \underline{F}\} & \text{Decomposition based on $FD_3: F \rightarrow EA$}\\
            R_5& : \{\underline{A}, C\} & \text{Decomposition based on $FD_5: A \rightarrow C$}\\
        \end{align*}
    \end{homeworkSubProblem}
\end{homeworkProblem}

\begin{homeworkProblem}
Based on Algorithm 16.6, first calculate the minimal cover.
    \begin{enumerate}[label=\textbf{Step {\arabic*}}, leftmargin=2cm]
        \item Consider the right side of the functional dependencies.
            The original functional dependencies is already right side minimal:
            \begin{itemize}
                \item $FG \rightarrow E$
                \item $HI \rightarrow E$
                \item $F \rightarrow G$
                \item $FE \rightarrow H$
                \item $H \rightarrow I$
            \end{itemize}
        \item Consider the left side of the functional dependencies.
            \begin{itemize}
                \item $FG \rightarrow E$ can be replaced with $F \rightarrow E$ because $F \rightarrow G$.
                \item $HI \rightarrow E$ can be replaced with $H \rightarrow E$ because $H \rightarrow I$.
                \item $FE \rightarrow H$ can be replaced with $F \rightarrow H$ because $F \rightarrow E$ (from previous conclusion).
            \end{itemize}
            So now the functional dependencies turn into:
            \begin{itemize}
                \item $F \rightarrow E$
                \item $H \rightarrow E$
                \item $F \rightarrow G$
                \item $F \rightarrow H$
                \item $H \rightarrow I$
            \end{itemize}
        \item Consider the redundancy of the functional dependencies.
            \begin{itemize}
                \item $F \rightarrow H$ and $H \rightarrow E$, so $F \rightarrow E$ is redundant.
            \end{itemize}
            So now the functional dependencies turn into minimal cover:
            $\displaystyle\left\{
                    \begin{array}{rcl}
                        H &\rightarrow& E \\
                        F &\rightarrow& G \\
                        F &\rightarrow& H \\
                        H &\rightarrow& I \\
                    \end{array}
            \right\}$

            $F$ is a candidate key of the relation $R$.
    \end{enumerate}

Then, for each left-hand-side X, form the relation:
        \begin{align*}
            R_1& : \{E, \underline{H}, I\} \\
            R_2& : \{\underline{F}, G, H\} \\
        \end{align*}

The key $F$ is contained in $R_2$ and there is no redundant relation.
Hence the result is the 3NF relations with dependency preservation and non-additive join property.
\end{homeworkProblem}
\end{document}
