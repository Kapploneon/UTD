\documentclass[]{article}

\usepackage[utf8]{inputenc}
\usepackage[english]{babel}
\usepackage{amsmath}
\usepackage{amsthm}
\usepackage{amssymb}
\usepackage{float}
\usepackage{listings}
\usepackage{titlesec}
\usepackage{color}
\usepackage{hyperref}
\usepackage{cleveref}

\titleformat{\subsection}{\normalfont\large\bfseries}{Exercise}{1em}{}

\definecolor{dkgreen}{rgb}{0,0.6,0}
\definecolor{gray}{rgb}{0.5,0.5,0.5}
\definecolor{mauve}{rgb}{0.58,0,0.82}
\definecolor{navyblue}{rgb}{0.0, 0.0, 0.35}

\lstset{
  % frame=tb,
  language=Java,
  aboveskip=3mm,
  belowskip=3mm,
  showstringspaces=false,
  columns=flexible,
  basicstyle={\ttfamily},
  numbers=left,
  numberstyle={\footnotesize\color{gray}\ttfamily},
  keywordstyle={\bfseries\color{navyblue}},
  commentstyle=\color{dkgreen},
  stringstyle=\color{mauve},
  breaklines=true,
  breakatwhitespace=true,
  tabsize=2,
  % frame=single,
  keepspaces=true,
}

\title{Homework 3}
\author{Hanlin He\footnote{\texttt{hxh160630@utdallas.edu}},
Lizhong Zhang\footnote{\texttt{lxz160730@utdallas.edu}}}
\date{\today}

\begin{document}

\maketitle

\section*{Chapter 5}

\subsection{54}

Since we have \lstinline{peek()} method for queue, we can simply let each thread
enqueue its thread ID after proposing the value, and then return value by which
the first thread in the queue (\lstinline{peek()} from queue) had proposed. Code
are shown as follow:
\begin{lstlisting}
public class QueueConsensus<T> extends ConsensusProtocol<T> { 
  Queue queue;
  public QueueConsensus() { 
    queue = new Queue();
  }
  // figure out which thread was first 
  public T decide(T value) {
    propose(value);
    int i = ThreadID.get();
    // enq current id.
    queue.enq(i)
    // return value of the first thread in the queue
    return proposed[queue.peek()];
  }
}
\end{lstlisting}
By doing so, we can allow any arbitrary large number of threads competing but
only one thread will be return from \lstinline{peek()}, and thus all threads
will decide on the same value. Therefore the consensus number would be infinite.

\subsection{57}

If we announced the thread’s value after dequeuing from the queue, it is
possible that the thread which `\lstinline{WIN}' propose its value after the
\lstinline{LOSE} thread trying to return the \lstinline{proposed[1-i]}, which
would lead to a null return value.

\subsection{58}

We can implement the binary consensus protocol using the StickyBit as follow:
\begin{lstlisting}
public class StickyBitConsensus { 
  StickyBit sb;
  public StickyBitConsensus() { 
    sb = new StickyBit();
  }
  public int decide(int value) {
    sb.write(value)
    // return value of the first thread performed the sb.write()
    return sb.read()
  }
}
\end{lstlisting}
This solution works because for any number of threads, only the first thread
successfuly performed \lstinline{sb.write()} would be able to change the value
of \lstinline{sb}. Then all thread will return the same value in \lstinline{sb}.

Given $m$ possible input, we can label all possible input with an integer from
$0$ to $m-1$. We can encode the integer in $\log_2m$ bit and then code with an
array of \lstinline{StickyBit} as follow:
\begin{lstlisting}
public class MStickyBitConsensus { 
  StickyBit[] sbits;
  public StickyBitConsensus(int m) {
    // suppose Math.log return ceiling of log.
    sbits = new StickyBit[Math.log(m)];
    for (int i = 0; i < sbits.length; i++)
      sbits[i] = new StickyBit();
  }

  public boolean decide(int value) {
    int i = sbits.length - 1;
    while (value != 0) {
      sbits[i].write(value & 1);
      // if the thread failed to write current StickyBit
      // there is no need to write to following StickyBits, 
      // because some other thread must have succeeded me.
      if (sbits[i].read != (value & 1))
        break;
      value >> 1;
      i--;
    }
    int ret = 0;
    for (StickyBit sb : sbits) {
      // if current StickyBit haven't been written yet, wait.
      while (sb.read() != 1 || sb.read() != 0);
      ret |= sb.read();
      ret << 1;
    }
    return ret;
  }
}
\end{lstlisting}

\section*{Chapter 9}

\subsection{104}

Suppose there are three thread $A$, $B$ and $C$, a possible execution would be:
\begin{enumerate}
  \item $A$ located the window $pred \rightarrow curr$.
  \item $B$ located the window $pred \rightarrow curr$, locked the window, added $x$ between $pred$ and $curr$, and then
  unlocked the window.
  \item $A$ locked the window, validation failed because $pred \nrightarrow
  curr$, unlocked the window and retried.
  \item $C$ located the window $pred \rightarrow x$, locked the window, removed
  $x$, and then unlocked the window.
  \item $A$ located the window $pred \rightarrow curr$ and previous steps
  happened again.
\end{enumerate}
From this execution, $A$ located the same window every time, but failed to move
forward because $B$ and $C$ added/removed an element between $pred$ and $curr$,
invalidating the windows located by $A$.

\subsection{110}

If setting the next field of the node to be deleted to \lstinline{null}, the
linked list would be broken. Any other threads trying to traverse the linked
list to a latter node would fail. Thus, it won't work by setting the next field
to \lstinline{null}.

\subsection{112}

The reasoning only conisdered the remove scenario. It is still possible that the
window $pred \rightarrow curr$ was modified by \lstinline{add(x)} to 
$pred \rightarrow x \rightarrow curr$. In this scenario, validation should still
fail because $pred \nrightarrow curr$.

\section*{Chapter 10}

\subsection{122}

It is necessary to hold the lock when checking that the queue is not empty.

Consider a queue with only one element. If we didn't lock the queue, it is
possible that two threads are both able to pass 
\lstinline{if (head.next == null)} check and proceed.

After the one thread \emph{lock-process-unlock} the queue, the other thread will
encounter \lstinline{NullPointerException} when it tried to dereference
\lstinline{head.next.value} in line 18 \lstinline{result = head.next.value;}

\subsection{125}

Based on the implementation:
\begin{itemize}
  \item The \lstinline{enq()} operation is wait-free/lock-free, because both
  operations in \lstinline{enq()} wait-free.
  \item The \lstinline{deq()} operation is not wait-free, but lock-free.
  
  A possible scenario would be, when a thread get the range by
  \lstinline{tail.get()}, some other threads get the same value and deque the
  value by invoking \lstinline{items[i].getAndSet(null);}. Then the prevois
  thread will fail because all element it encountered would be \lstinline{null}
  and the thread would retry.

  A thread could be overtaken by other threads as described above arbitrary
  number of times. Hence the \lstinline{deq()} is not wait-free.

  However, as long as there are elements in the queue, some threads will always
  succeed in dequeue operation, since there are always some not-null value
  returned from \lstinline{items[i].getAndSet(null)}. Therefore,
  \lstinline{deq()} is lock-free. 
\end{itemize}

Furthermore, the linearization point for \lstinline{deq()} method would be the
first time that \lstinline{items[i].getAndSet(null)} return a not-null value.

The linearization point for \lstinline{enq()} depends on the execution order of
\lstinline{enq()} and \lstinline{deq()}:
\begin{itemize}
\item If all \lstinline{enq()} invocations happened before all
\lstinline{deq()}, then the linearization point would be line 6:
\begin{lstlisting}[firstnumber=6]
int i = tail.getAndIncrement();\end{lstlisting}
Because the order of the elements dequeued would be solely depend on the
index $i$ returned from \lstinline{tail.getAndIncrement()}.

\item If some \lstinline{deq()} invocations happens concurrently with the
\lstinline{enq()}, actual linearization point could be line 7:
\begin{lstlisting}[firstnumber=7]
items[i].set(x);\end{lstlisting}
Because, \lstinline{deq()} invocation could happen before all \lstinline{enq()}
finished setting the element in the array. During such scenario, the
\lstinline{deq()} will return the first not-null element it encountered.

A concrete example would be:
\begin{enumerate}
  \item Thread $A$ tried to \lstinline{enq(1)}, while thread $B$ tried to
  \lstinline{enq(2)}.
  \item Thread $A$ got \lstinline{i = 10}, while thread $B$ got
  \lstinline{i = 11}.
  \item Thread $B$ successfuly invoked \lstinline{item[12].set(2)} before $A$.
  \item Before thread $A$ invoke \lstinline{item[10].set(1)}, another thread $C$
  invoked \lstinline{deq()}.
  \item The first element thread $C$ encountered would be \lstinline{item[12]},
  which is $2$.
  \item \lstinline{C.deq()} would return $2$.
\end{enumerate}
Therefore the linearization point should be line 7.

\end{itemize}

\end{document}
