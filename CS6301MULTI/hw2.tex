% !TEX TS-program = XeLaTeX
\documentclass[]{article}

\usepackage[utf8]{inputenc}
\usepackage[english]{babel}

\usepackage{fontspec}
\setmonofont[
  Contextuals={Alternate}
]{Fira Code}

\makeatletter
\def\verbatim@nolig@list{}
\makeatother

\usepackage{amsmath}
\usepackage{amsthm}
\usepackage{amssymb}
\usepackage{float}
\usepackage{booktabs}
\usepackage{listings}
\usepackage{titlesec}
\usepackage{listings}
\usepackage{color}
\usepackage{hyperref}
\usepackage{cleveref}

\titleformat{\subsection}{\normalfont\large\bfseries}{Exercise}{1em}{}

\definecolor{dkgreen}{rgb}{0,0.6,0}
\definecolor{gray}{rgb}{0.5,0.5,0.5}
\definecolor{mauve}{rgb}{0.58,0,0.82}
\definecolor{navyblue}{rgb}{0.0, 0.0, 0.35}

\lstset{frame=tb,
  language=Java,
  aboveskip=3mm,
  belowskip=3mm,
  showstringspaces=false,
  columns=flexible,
  basicstyle={\footnotesize\ttfamily},
  numbers=left,
  numberstyle={\footnotesize\color{gray}\ttfamily},
  keywordstyle={\bfseries\color{navyblue}},
  commentstyle=\color{dkgreen},
  stringstyle=\color{mauve},
  breaklines=true,
  breakatwhitespace=true,
  tabsize=3,
  frame=single,
  keepspaces=true,
}

\title{Homework 2}
\author{Hanlin He (hxh160630)}
\date{\today}

\begin{document}

\maketitle

\section*{Chapter 7}

\subsection{85}

Consider a thread acquired and release the lock for the first time. Let $qnodeX$
be the node created for the thread. After the thread released the lock, we will
have:
\begin{itemize}
    \item $tail = qnodeX$
    \item $qnodeX.locked = false$
\end{itemize}
If the same thread tries to acquire the lock again, the execution of
\lstinline{lock} method would be:
\begin{lstlisting}
public void lock() {
    Qnode qnode = myNode.get();         // qnode <- qnodeX
    qnode.locked = true;                // qnodeX.locked <- true
    Qnode pred = tail.getAndSet(qnode); // pred <- qnodeX
    while (pred.locked) {}      // get stuck in pred.locked forever
}
\end{lstlisting}
At line 4, return tail from \lstinline{tail.getAndSet(qnode)} will return $qnodeX$
itself. And the $locked$ value was already set to $ true$, so the
\lstinline{while}
loop will never break.

\subsection{91}

\begin{itemize}
    \item \verb|testAndSet()| spin lock:
\begin{lstlisting}
public boolean isLocked() {
    return state.get();
}
\end{lstlisting}
    \item CLH queue lock:
\begin{lstlisting}
public boolean isLocked() {
    return tail.get().locked;
}
\end{lstlisting}
    \item MCS queue lock:
\begin{lstlisting}
public boolean isLocked() {
    return tail.get() != null;
}
\end{lstlisting}
\end{itemize}

\section*{Chapter 8}

\subsection{95}
\lstinputlisting{SavingsAccount/src/Account.java}
\lstinputlisting{SavingsAccount/src/SavingsAccount.java}
\lstinputlisting{SavingsAccount/src/SavingsAccountWithPreferredWithdraw.java}
\lstinputlisting{SavingsAccount/src/SavingsAccountWithTransfer.java}

\section*{Chapter 8}
\subsection{24}
\subsubsection*{For the first history}

Let $H_1$ denotes the history, $\alpha$ denotes the
operation $A:r.\mathrm{read}(1)$, $\beta$ and $\gamma$ denote the operations
$B:r.\mathrm{write}(1)$ and $B:r.\mathrm{read}(2)$, and $\delta$ denotes the
operation $C:r.\mathrm{write}(2)$:

\begin{itemize}
    \item Quiescently Consistent: Yes.

        After execution of $\alpha$, $\beta$ and $\gamma$, the program became
        quiescent. Thus, the history would be quiescently consistent if the
        result of $\delta$ appear after $\alpha$, $\beta$ and $\gamma$. Suppose
        $\gamma$ was the last operation to take effect among $\alpha$, $\beta$
        and $\gamma$, the history would be legal and quiescently consistent.

    \item Sequentially Consistent: Yes.

        Consider the sequential history shown in \cref{24-1-seq}.
        \begin{align}
            \label{24-1-seq}
            S_1 \equiv \beta \rightarrow \alpha \rightarrow \delta \rightarrow \gamma
        \end{align}

        $S_1$ is apparently equivalent with $H_1$. And the object subhistory of
        $r$ is legal. Thus, the history $H_1$ is sequentially consistent.

    \item Linearizable: Yes.

        Consider the sequential history in \cref{24-1-seq} again. The preceding
        relations of $H_1$ is
        \begin{align}
            \rightarrow_{H_1} \equiv \left\{ 
                \alpha \rightarrow \gamma,
                \beta \rightarrow \gamma,
                \delta \rightarrow \gamma,
            \right\}
        \end{align}
        Apparently, $\rightarrow_{H_1} \subseteq \rightarrow_{S_1}$. Thus, the
        history $H$ is linearizable.
\end{itemize}

\subsubsection*{For the second history}

Let $H_2$ denotes the history, $\alpha$ denotes the
operation $A:r.\mathrm{read}(1)$, $\beta$ and $\gamma$ denote the operations
$B:r.\mathrm{write}(1)$ and $B:r.\mathrm{read}(1)$, and $\delta$ denotes the
operation $C:r.\mathrm{write}(2)$:

\begin{itemize}
    \item Quiescently Consistent: Yes.

        After execution of $\alpha$, $\beta$ and $\gamma$, the program became
        quiescent. Thus, the history would be quiescently consistent if the
        result of $\delta$ appear after $\alpha$, $\beta$ and $\gamma$. Suppose
        $\beta$ was the last operation to take effect among $\alpha$, $\beta$
        and $\gamma$, the history would be legal and quiescently consistent.

    \item Sequentially Consistent: Yes.

        Consider the sequential history $S_2$ shown in \cref{24-2-seq}.
        \begin{align}
            \label{24-2-seq}
            S_2 \equiv \delta \rightarrow \beta \rightarrow \alpha \rightarrow \gamma
        \end{align}

        $S_2$ is apparently equivalent with $H_2$. And the object subhistory of
        $r$ is legal. Thus, the history $H_2$ is sequentially consistent.

    \item Linearizable: Yes.

        Consider the sequential history in \cref{24-2-seq} again. The preceding
        relations of $H_2$ is
        \begin{align}
            \rightarrow_{H_2} \equiv \left\{ 
                \alpha \rightarrow \gamma,
                \beta \rightarrow \gamma,
                \delta \rightarrow \gamma,
            \right\}
        \end{align}
        Apparently, $\rightarrow_{H_2} \subseteq \rightarrow_{S_2}$. Thus, the
        history $H_2$ is linearizable.
\end{itemize}

\subsection{27}

Since Java does not guarantee linearizability, the definition of buffer would
lead to linearizability problems.
\begin{lstlisting}
T[] items = (T[]) new Object[Integer.MAX_VALUE];
\end{lstlisting}
The \lstinline{items} was defined as an array of generic type \lstinline{T},
instead of an array of \lstinline{AtomicReference<T>} or simply
\lstinline{AtomicReferenceArray<T>}. The changes of one element in
\lstinline{items} within one thread \lstinline{items[slot] = x;} may not be
noticeable to other threads immediately. So it is possible to have two threads
$T_1$ and $T_2$ with a history $H$ that:
\begin{align}
    \label{27-H}
    H: T_1.enq(1) \rightarrow T_2.deq() \text{ throw empty exception}
\end{align}
because \lstinline{value = items[slot]} in $T_2$ was still \lstinline{null}
after $T_1.enq(1)$. However, the legal sequence history for \cref{27-H} is:
\begin{align}
    \label{27-S}
    S: T_2.deq() \text{ throw empty exception} \rightarrow T_1.enq(1)
\end{align}

Apparently, $\rightarrow_H \nsubseteq \rightarrow_S$, the implementation is not
linearizable.

\end{document}
