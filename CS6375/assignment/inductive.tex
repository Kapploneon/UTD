\documentclass[10pt]{article}

\usepackage[l2tabu, orthodox]{nag}
\usepackage{tabu}
\usepackage[onehalfspacing]{setspace}
\usepackage{amsmath}
\usepackage{amsthm}
\usepackage{amsfonts}
\theoremstyle{definition}
\newtheorem{definition}{Definition}

\usepackage{float}
\usepackage{enumitem}

% For plot
\usepackage{pgfplots}
\usepackage{pgfplotstable}
\usepackage{booktabs}
\usepackage{array}
\usepackage{colortbl}
% \pgfplotstableset{% global config, for example in the preamble
%   every head row/.style={before row=\toprule,after row=\midrule},
%   every last row/.style={after row=\bottomrule},
%   fixed,precision=2,
% }

\usepackage{subcaption}

% As a rule of thumb it should be loaded at the end of the preamble, after all
% the other packages. A few exceptions exist, such as the cleveref package that
% is also mentioned in this post. Hence, cleveref should be loaded after
% hyperref.
\usepackage{hyperref}

% This package introduces the \cref command. When using this command to make
% cross-references, instead of \ref or \eqref, a word is placed in front of the
% reference according to the type of reference: fig. for figures, eq. for
% equations
\usepackage{cleveref}

\title{Inductive Learning}
\author{Hanlin He\footnote{hxh160630@utdallas.edu}}

\begin{document}
\maketitle

\section{}
\begin{align}
    J &=\frac{1}{2m}\sum_{i=1}^{i=m}(h_\theta(x^{(i)})-y^{(i)})^2 \\
      &=\frac{1}{2\times{4}}\left((\theta_0+3\theta_1-2)^2+(\theta_0+\theta_1-2)^2+(\theta_0-1)^2+(\theta_0+4\theta_1-3)^2\right)\\
      &=\frac{1}{8}(4\theta_0^2+26\theta_1^2+16\theta_0\theta_1-14\theta_0-40\theta_1+18)
\end{align}
Let $\theta_0^i$ denotes the $i$th round value of $\theta_0$.
\begin{itemize}
    \item Round 0
        \begin{align}
            \theta_0^0 &= 0\\
            \theta_1^0 &= 1
        \end{align}
    \item Round 1
        \begin{align}
            \theta_0^1&=\theta_0^0-\alpha\frac{\partial}{\partial\theta_0}J(\theta_0,\theta_1)\\
            \theta_1^1&=\theta_1^0-\alpha\frac{\partial}{\partial\theta_1}J(\theta_0,\theta_1)
        \end{align}
\end{itemize}

\section{}
\begin{itemize}
    \item False Positive: $20\%$
    \item False Negative: $10\%$
\end{itemize}

\section{}

\section{}

\begin{definition}
    A hypothesis $h$ is \emph{consistent} with a set of training examples D if
    and only if $h(x) = c(x)$ for each example $\langle x, c(x)\rangle$ in $D$.
    \[Consistent(h,D)\equiv(\forall\langle x,c(x)\rangle\in D) h(x)=c(x)\]
\end{definition}

\begin{definition}
    The \emph{version space}, denoted $VS_{H,D}$, with respect to hypothesis
    space $H$ and training examples $D$, is the subset of hypotheses from $H$
    consistent with the training examples in $D$.
    \[VS_{H,D}\equiv\{h\in H | Consistent(h,D)\}\]
    \end{definition}

\section{}
The most general hypothesis has (\textbf{?}) value for each attribute.

\section{}
\begin{enumerate}[label={\alph*)},font={\color{red!50!black}\bfseries}]
    \item $|X| = 3^4 = 81$.
    \item $(instance) = 2^{16}$.
    \item 
\end{enumerate}

\section{}

\begin{align*}
    h_0 &\leftarrow \langle \emptyset, \emptyset, \emptyset, \emptyset,
    \emptyset \rangle \\
    h_1 &\leftarrow \langle 1, 1, 0, 1, 1 \rangle \\
    h_2 &\leftarrow \langle 1, 1, 0, 1, 1 \rangle \\
    h_3 &\leftarrow \langle 1, 1, ?, 1, ? \rangle \\
    h_4 &\leftarrow \langle 1, 1, ?, 1, ? \rangle \\
    h_5 &\leftarrow \langle 1, 1, ?, 1, ? \rangle
\end{align*}

\section{}
\section{}
\begin{tikzpicture}
    \begin{axis}[xmin=0,xmax=10,ymin=0,ymax=10,enlargelimits=false,
    minor x tick num=1,minor y tick num=1,]
    \addplot+[only marks,color=red,mark=+,mark size=3pt]
    coordinates {(4,4) (5,3) (6,5)};
    \addplot+[only marks,color=blue,mark=-,mark size=3] 
    coordinates {(1,3) (2,6) (5,1) (5,8) (9,4)};
    \addplot[red!25,forget plot,dashed] 
    coordinates {(4,5) (4,3) (6,3) (6,5) (4,5)};
    \addplot[blue!25,forget plot,dashed] 
    coordinates {(2,8) (2,1) (9,1) (9,8) (2,8)};
    \end{axis}
\end{tikzpicture}
\section{}

\end{document}
