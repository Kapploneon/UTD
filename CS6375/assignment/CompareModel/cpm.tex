\documentclass[12pt, letterpaper]{article}

\usepackage[l2tabu, orthodox]{nag}
\usepackage{tabu}
\usepackage{amsmath}
\usepackage{amsthm}
\usepackage{amsfonts}
\usepackage{float}
\usepackage{graphicx}
\usepackage{color}
\usepackage{standalone}

\usepackage{parskip}
% As a rule of thumb it should be loaded at the end of the preamble, after all
% the other packages. A few exceptions exist, such as the cleveref package that
% is also mentioned in this post. Hence, cleveref should be loaded after
% hyperref.
\usepackage{hyperref}
\definecolor{linkcolour}{rgb}{0.8,0.2,0.5}
\hypersetup{colorlinks,breaklinks,urlcolor=linkcolour, linkcolor=linkcolour}

% This package introduces the \cref command. When using this command to make
% cross-references, instead of \ref or \eqref, a word is placed in front of the
% reference according to the type of reference: fig. for figures, eq. for
% equations
\usepackage{cleveref}

\title{Assignment 4 Part I}
\author{Hanlin He\footnote{hxh160630@utdallas.edu},
Tao Wang\footnote{txw162630@utdallas.edu}}

\begin{document}
\maketitle

\section{}

\begin{align*}
E_{agg}(x) &= E\left[\bigg\{\frac{1}{M}\sum_{i=1}^M\epsilon_i(x)\bigg\}^2\right]\\
&= \frac{1}{M^2}E\left[\bigg\{\sum_{i=1}^M\epsilon_i(x)\bigg\}^2\right]\\
&= \frac{1}{M^2}E\bigg[\sum_{i=1}^M\epsilon_i^2(x)+\sum_{i=1}^M\sum_{j=1}^M\epsilon_i(x)\epsilon_j(x)\bigg]\qquad\text{// }\forall i\neq j,E[\epsilon_i(x)\epsilon_j(x)]=0\\
&= \frac{1}{M^2}E\bigg[\sum_{i=1}^M\epsilon_i^2(x)\bigg]\\
&= \frac{1}{M^2}\sum_{i=1}^ME[\epsilon_i^2(x)]\qquad\text{// }E_{avg}=\frac{1}{M}\sum_{i=1}^ME[\epsilon_i^2(x)]\\
&= \frac{1}{M}E_{avg}
\end{align*}

\section{}

Since all \emph{convex} function have following inequality:
\[f\bigg(\sum_{i=1}^M\lambda_ix_i\bigg)\leq\sum_{i=1}^M\lambda_if(x_i)\]

Given that $f: x \rightarrow x^2$ is a convex function.

\begin{align*}
E_{agg}(x) &= E\left[\bigg\{\frac{1}{M}\sum_{i=1}^M\epsilon_i(x)\bigg\}^2\right]\\
&= E\left[\bigg\{\sum_{i=1}^M\frac{1}{M}\epsilon_i(x)\bigg\}^2\right]\\
&\leq E\left[\sum_{i=1}^M\frac{1}{M}\bigg\{\epsilon_i(x)\bigg\}^2\right]\\
&= \frac{1}{M}E\left[\sum_{i=1}^M\bigg\{\epsilon_i(x)\bigg\}^2\right]\\
&= E_{avg}
\end{align*}

\end{document}
