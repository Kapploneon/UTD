\begin{homeworkProblem}[Bounding Recurrences]

\begin{homeworkSubProblem}[\texorpdfstring{$T(n)=2T(n/2)+n^4$}{T(n)=2T(n/2)+n\textasciicircum 4}]\label{sub6.1}
$T(n)=2T(n/2)+n^4$ indicates that at the $i$ recursive level, the total
operations are:

\begin{equation}
f_i(n) = 2^i\times\left(\frac{n}{2^i}\right)^4=\frac{n^4}{2^{3i}}
\end{equation}

The depth of the recursion tree is $\lg n$, hence, the total running time is:

\begin{equation}T(n) =\sum_{i=0}^{\lg n}f_i(n) = \sum_{i=0}^{\lg n}\frac{n^4}{2^{3i}}
\end{equation}

Now consider the ratio of successive level sums:

\begin{equation}
r = \frac{f_{i+1}(n)}{f_i(n)}
  = \cfrac{\cfrac{n^4}{2^{3(i+1)}}}{\cfrac{n^4}{2^{3i}}}
  = \frac{1}{8}
  < 1
\end{equation}

Which means the total running time $T(n)$ is mainly decided by the $T(root)$, i.e.

\begin{equation}
T(n) = \Theta(n^4)
\end{equation}

\end{homeworkSubProblem}

\begin{homeworkSubProblem}[\texorpdfstring{$T(n)=16T(n/4)+n^2$}{T(n)=16T(n/4)+n\textasciicircum 2}]

$T(n)=16T(n/4)+n^2$ indicates that at the $i$ recursive level, the total
operations are:

\begin{equation}
f_i(n) = 16^i*\left(\frac{n}{4^i}\right)^2=n^2
\end{equation}

which is not related to the level $i$. As the depth of the recursion tree is $\log_4 n$, hence, the total running time is:

\begin{equation} \label{eq:6.2.1}
T(n)
=\sum_{i=0}^{\log_4 n}f_i(n)
= \sum_{i=0}^{\log_4 n}n^2
= n^2 \log_4 n
= \Theta(n^2\log n)
\end{equation}

\end{homeworkSubProblem}

\begin{homeworkSubProblem}[\texorpdfstring{$T(n)=2T(n/3)+T(n/4)+n$}{T(n)=2T(n/3)+T(n/4)+n}]

Assume $T(n)=2T(n/3)+T(n/4)+n$ indicates that at the $i$ recursive level,
the total operations are $f_i(n)$. We can write the $f_i(n)$ as:

\begin{equation}
\begin{split}
f_0(n) & = n ,\\
f_1(n) & = \left(\frac{2}{3}+\frac{1}{4}\right) n
         = \left(\frac{11}{12}\right) n ,\\
f_2(n) & = \left(\frac{4}{3^2}+\frac{4}{3\times 4}+\frac{1}{4^2}\right) n
         = \left(\frac{121}{144}\right) n
         = \left(\frac{11}{12}\right)^2 n ,\\
f_3(n) & = \left(\frac{8}{3^3}+\frac{12}{3^2\times 4}+\frac{6}{3\times 4^2}
            +\frac{1}{4^3}\right) n
         = \left(\frac{1331}{1728}\right) n
         = \left(\frac{11}{12}\right)^3 n ,\\
\dotso \\ %“other dots”
%\xrightarrow[1]{aaa} f_i(n) & = \\
\end{split}
\end{equation}

According the pattern of $f_i(n)$ above, it is easy to conclude and prove
by induction that the general form of $f_i(n)$ is:

\begin{equation}
f_i(n) = \left(\frac{11}{12}\right)^i n
\end{equation}

Now consider the ratio of successive level sums:

\begin{equation}
r = \frac{f_{i+1}(n)}{f_i(n)}
  = \cfrac{\left(\cfrac{11}{12}\right)^{i+1} n}{\left(\cfrac{11}{12}\right)^i n}
  = \frac{11}{12}
  < 1
\end{equation}

Which means that the total running time $T(n)$ is mainly decided by the $T(root)$, i.e.

\begin{equation} \label{eq:6.3}
T(n) = \Theta(n)
\end{equation}

\end{homeworkSubProblem}

\begin{homeworkSubProblem}[\texorpdfstring{$T(n)=T(n-1)+\sqrt{n}$}{T(n)=T(n-1)+sqrt(n)}]

$T(n) = T(n-1) + \sqrt{n}$ indicates that at the $i$ recursive level, the total operations are:

\begin{equation}
f_i(n) = \sqrt{n-i}
\end{equation}

The depth of the recursion tree is $n$, hence, the total running time is:

\[T(n) =\sum_{i=0}^n f_i(n) = \sum_{i=0}^n \sqrt{n-i} = \sum_{i=1}^n \sqrt n\]

Now consider the ratio of successive level sums:

\begin{equation}
r = \frac{f_{i+1}(n)}{f_i(n)}
  = \frac{\sqrt{n-i+1}}{\sqrt{n-i}}
  < 1
\end{equation}

Which means that the total running time $T(n)$ is mainly decided by the $T(root)$, i.e.

\begin{equation} \label{eq:6.4}
T(n) = \Theta(\sqrt n)
\end{equation}

\end{homeworkSubProblem}

\begin{homeworkSubProblem}[\texorpdfstring{$T(n)=3T(n/2)+5n$}{T(n)=3T(n/2)+5n}]
$T(n)=3T(n/2)+5n$ indicates that at the $i$ recursive level, the total
operations are:
\begin{equation}
f_i(n) = 3^i\times\left(\frac{5n}{2^i}\right)
       = \left(\frac{3}{2}\right)^i 5n
\end{equation}

The depth of the recursion tree is $\lg n$, hence, the total running time is:

\begin{equation}
T(n) =\sum_{i=0}^{\lg n}\left(\frac{3}{2}\right)^i 5n
     =5n\sum_{i=0}^{\lg n}\left(\frac{3}{2}\right)^i
\end{equation}
Now consider the ratio of successive level sums:

\begin{equation}
r = \frac{f_{i+1}(n)}{f_i(n)}
  = \cfrac{\left(\cfrac{3}{2}\right)^{i+1} 5n}{\left(\cfrac{3}{2}\right)^i 5n}
  = \frac{3}{2}
  > 1
\end{equation}

Which means the total running time $T(n)$ is mainly decided by the $T(leaf)$, the last recursion's running time i.e.

\begin{equation}
T(n) = 5 \times 3^{\lg n}= \Theta(n^{\lg3})
\end{equation}

\end{homeworkSubProblem}

\begin{homeworkSubProblem}[\texorpdfstring{$T(n)=T(\sqrt{n})+7$}{T(n)=T(sqrt(n))+7}]

$T(n)=T(\sqrt{n})+7$ indicates that at the $i$ recursive level, the total operations are:

\begin{equation}
f_i(n) = 7
\end{equation}

The depth of the recursion tree is $n$, hence, the total running time is:

\begin{equation}
T(n) =\sum_{i=0}^{\sqrt{n}} f_i(n) = \sum_{i=0}^{\sqrt{n}} 7
\end{equation}

In this case, the ratio of successive level sums is apparently $r = 1$, which means that the total running time $T(n)$ is mainly decided by the depth of the recursion tree.

Assume the base case is $n=1$, $\sqrt[2i]{n}$, in which $i$ is a natural number representing the $ith$ recursion, will not reach $1$, though $\lim_{i=0}^{\infty} \sqrt[2i]{n}= 1$. Thus, the recursion will not reach base case, i.e.

\begin{equation} \label{eq:6.6}
T(n) = \Theta(\infty)
\end{equation}

\end{homeworkSubProblem}

\end{homeworkProblem}
