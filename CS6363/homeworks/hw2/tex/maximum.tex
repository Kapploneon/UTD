\begin{homeworkProblem}[Maximum Subarray Sum]

Let $maxSum(i, j)$ be the maximum sub-array sum in $A[i \ldots j]$, there are two cases:
\begin{itemize}
    \item $maxSum(i, j)$ does not include $A[j]$, then
        \[maxSum(i,j) = maxSum(i, j-1)\]
    \item $maxSum(i, j)$ does include $A[j]$, then
        \[maxSum(i,j) = maxEndAt(i, j)\]
    in which $maxEndAt(i, j)$ is the max sub-array sum in $A[i \ldots j]$ restricted to include $A[j]$.
\end{itemize}

There are again two cases for $maxEndAt(i, j)$:
\begin{itemize}
    \item $maxEndAt(i, j)$ is the element $A[j]$ itself, i.e.
        \[maxEndAt(i, j) = A[j]\]
    \item $maxEndAt(i, j)$ is the element $A[j]$ plus $maxEndAt(i, j-1)$, i.e.
        \[maxEndAt(i, j) = A[j] +maxEndAt(i, j-1)\]
\end{itemize}

Thus, the recursive algorithm solving the maximum sub-array sum can be described as
\cref{recursive_max_sub_sum}.

\begin{algorithm}[H]
    \caption{Recursive Solution to Maximum Sub-array Sum} \label{recursive_max_sub_sum}
    \begin{algorithmic}[1]
        \Procedure{MaxSum}{$A[i\ldots j]$}
            \If{$i>j$}
                \State \textbf{return } $0$
            \EndIf
            \State \textbf{return }$\max\left\{\text{\textsc{MaxEndAt}}(A[i\ldots j])\text{,\ }\text{\textsc{MaxSum}}(A[i\ldots j-1])\right\}$
        \EndProcedure
        \\\hrulefill
        \Procedure{MaxEndAt}{$A[i\ldots j]$}
            \If{$i>j$}
                \State \textbf{return } $0$
            \EndIf
            \State \textbf{return } $\max\left\{A[j]\text{,\ }A[j] + \text{\textsc{MaxEndAt}}(A[i\ldots j-1]) \right\}$
        \EndProcedure
    \end{algorithmic}
\end{algorithm}

To find the maximum sub-array sum of $A[1\ldots n]$, we can call \textsc{MaxSum}$(A[1\ldots n])$.

\pagebreak

Observing \cref{recursive_max_sub_sum}, we can find that:
\begin{itemize}
    \item In \textsc{MaxEndAt}, each recursion is strictly based on the previous call,
        \\i.e. \textsc{MaxEndAt}$(i, j)$ is strictly based on \textsc{MaxEndAt}$(i, j-1)$.
    \item In \textsc{MaxSum}, each recursion is strictly based on the previous call and \textsc{MaxEndAt}, \\i.e. \textsc{MaxSum}$(i, j)$ is strictly based on \textsc{MaxSum}$(i, j-1)$ and \textsc{MaxEndAt}$(i, j)$.
\end{itemize}

Therefore, we can memorize \textsc{MaxSum} and \textsc{MaxEndAt} in each recursion.
The memoized algorithm is shown in \cref{memoized_max_sub_sum}.

Globally defined array $R[1 \ldots n]$ and $M[i \ldots n]$.

\begin{algorithm}[H]
    \caption{Memoized Solution to Maximum Sub-array Sum}\label{memoized_max_sub_sum}
    \begin{algorithmic}[1]
        \Procedure{MemMaxSum}{$A[i\ldots j]$}
            \If{$i>j$}
                \State \textbf{return } $0$
            \EndIf
            \If{$R[j]$ undefined}
            \State $R[j] = \max\left\{\text{\textsc{MemMaxEndAt}}(A[i\ldots j])\text{,\ }\text{\textsc{MemMaxSum}}(A[i\ldots j-1])\right\}$
            \EndIf
            \State \textbf{return } $R[j]$
        \EndProcedure
        \\\hrulefill
        \Procedure{MemMaxEndAt}{$A[i\ldots j]$}
            \If{$i>j$}
                \State \textbf{return } $0$
            \EndIf
            \If{$M[j]$ undefined}
            \State $M[j] = \max\left\{A[j]\text{,\ }A[j] + \text{\textsc{MemMaxEndAt}}(A[i\ldots j-1])\right\}$
            \EndIf
            \State \textbf{return } $M[j]$
        \EndProcedure
    \end{algorithmic}
\end{algorithm}

To find the maximum sub-array sum of $A[1\ldots n]$, we can call \textsc{MemMaxSum}$(A[1\ldots n])$.

\pagebreak
\noindent
\textbf{Applying DP:\ }
According to the previous observation, \textsc{MemMaxSum}$(A[1\ldots n])$ depends on two array
$R[1\ldots n]$ and $M[1\ldots n]$, each ranging over $\mathcal{O}(n)$ values, and is strictly
based on previous call. Hence the above recursive algorithm can be turned into a DP
algorithm using this two array. The array can be filled starting at $1$ and going up to $n$.
First the $M[1\ldots n]$, then the $R[1\dots n]$.

The dynamic programming algorithm is shown in \cref{dp_max_sub_sum}.

\begin{algorithm}[H]
    \caption{Dynamic Programming Solution to Maximum Sub-array Sum}\label{dp_max_sub_sum}
    \begin{algorithmic}[1]
        \Procedure{DPMaxSum}{$A[1\ldots n]$}
            \State Let $R[0\ldots n]$ and $M[0\ldots n]$ be an array.
            \State $M[0] = 0$, $R[0] = 0$
            \For{$i = 1$ to $n$}
                \State $M[i] = \max\left\{A[i]\text{,\ }A[i] + M[i-1])\right\}$
            \EndFor
            \For{$i = 1$ to $n$}
                \State $R[j] = \max\left\{M[i]\text{,\ }R[i-1])\right\}$
            \EndFor
            \State \textbf{return }$R[n]$
        \EndProcedure
    \end{algorithmic}
\end{algorithm}

With two for loop of $n$ iteration and constant time operations in each iteration, the running time for \cref{dp_max_sub_sum} is $\mathcal{O}(n)$.

\subsection{Professor's Solution}

\begin{algorithm}[H]
    \caption{Professor's Solution to Maximum Sub-array Sum}\label{recursive_max_sub_sum_pro}
    \begin{algorithmic}
        \Procedure{recSum}{$b, index$}
            \If{$index > n$}
                \Return $0$
            \EndIf
            \If{$b = 1$}
                \Return $\max\{0, A[index]+\ProcedureName{recSum}{1,index+1}\}$
            \EndIf
            \Return $\max\{\ProcedureName{recSum}{0,index+1}, A[index]+\ProcedureName{recSum}{1,index+1}\}$
        \EndProcedure
    \end{algorithmic}
\end{algorithm}

Meaning:

\begin{itemize}
    \item \ProcedureName{recSum}{0,index} is max sub-array sum in $A[index\ldots n]$.
    \item \ProcedureName{recSum}{1,index} is max of $0$ or max sub-array sum where
        first element is $A[index]$.
\end{itemize}

$b$ is used to tell next recursion whether the previous element is taken.
If $b=1$, then the previous element is taken.

To get the maximum sub-array sum of $A[1 \ldots n]$, call \ProcedureName{recSum}{0,1}.


\end{homeworkProblem}
