\section{Greedy Algorithm}

\subsection{Greedy in a Glance}

Greedy Algorithm is repeatedly taking locally optimal step in
hopes reaching global optimal solution.

Often Greedy is WRONG!

So we should assume greedy is wrong, unless proven otherwise.

However, if we can prove greedy works in some circumstance,
it would be a simple and fast solution in most cases.

\subsection{Greedy Class Scheduling}
\subsubsection{Description of Problem}

Target: On given day of week, we want to take as many classes as possible.
Constraint: We cannot take two classes whose times overlap.

\AlgoInput Given $S[1 \ldots n]$ and $F[1 \ldots n]$, which 
\[S[i] = \text{start time of class} i \]
\[F[i] = \text{finish time of class} i \]

\AlgoOutput Select largest subset $X \subseteq \{1 \ldots n \}$,
s.t. for $i \neq j \in X$ either $S[i] > F[j]$ or $S[j] > F[i]$.

\subsubsection{Analysis}
Recursive approach: consider class 1.
\[B_4 = \{ i | 2 \leq i \leq n \text{ and } F[i] < S[1] \}\]
\[L_8 = \{ i | 2 \leq i \leq n \text{ and } S[i] > F[1] \}\]

DP approach: \bigO{n} running time.

But we can do better with greedy: Pick class which finishes first.

In other word, the algorithm is:
\begin{quote}
    Scan classes in increasing order of finishing time.
    Each time encounter non-conflicting class, select it.
\end{quote}
as described in \cref{greedy_class_scheduling}.

\begin{algorithm}[H]
    \caption{Greedy Solution for Class Scheduling}\label{greedy_class_scheduling}
    \begin{algorithmic}[1]
        \Procedure{GreedySchedule}{$S[1 \ldots n]$, $F[1 \ldots n]$}
            \State Sort $F$ and permute $S$ to match.
            \State $count = 1$
            \State $X[count] = 1$ \Comment{$X[1 \ldots n]$}
            \For{$i = 2 \text{ to } n$}
                \If{$S[i] > F[X[count]]$}
                    \State $count = count + 1$
                    \State $X[count] = i$
                \EndIf
            \EndFor
            \Return $X[1 \ldots count]$
        \EndProcedure
    \end{algorithmic}
\end{algorithm}

Running time: $\Theta(n \log n) \rightarrow $ sorting running time.

\subsubsection{Proof}
Note that there may be many optimal solutions.
And Greedy produces unique solution.
So we cannot argue any optimal solution looks like greedy.

\begin{lemma}
At least one optimal solution include class that finishes first.
\end{lemma}

\begin{proof}
    Let $f$ be class that finishes first, $X$ be an optimal set of classes.
    If $f \in X$ then lemma proven.
    Otherwise, $f \notin X$. Let $g$ be class in $X$ that finishes first.
    Since $f$ finishes before $g$, $f$ cannot conflict with any class in $X \setminus \{g\}$

    Thus $X^\prime = X \cup \{f\} \setminus \{g\}$ is a valid solution with max size and contain $f$.
\end{proof}

\begin{theorem}
    Greedy schedule is optimal.\footnote{Can be proved using induction}
\end{theorem}

\begin{proof}
    Let $f$ be class that finishes first. $L$ is set that starts after $f$ finishes.
    Best schedule containing $f$ is optimal. This best schedule must be optimal on $L$.
    $L$ is strictly smaller set of classes, so can apply induction.
\end{proof}

\subsection{General Prove Method for Greedy Algorithm: Exchange Argument}
\begin{enumerate}
    \item Assume there is a optimal solution other than greedy algorithm.
    \item Fine the ``first'' difference between it and greedy algorithm.
    \item Argue can exchange optimal choice for greedy one, without degrading solution value.
\end{enumerate}
