\section{Basic}
\subsection{What is an algorithm?}
Unambiguous, mechanically executable sequence of elementary operations.

There are certain types of algorithm:

\begin{tabular}{ ll }
Traditional (This course’s main focus.) & Modern algorithm research\\
\hline
Deterministic & Randomized \\
Exact & Approximate\\
Off-line & On-line\\
Sequential & Parallel\\
\end{tabular}

\subsection{Input \& Output}

View algorithm as a function with well defined inputs mapping to specific
outputs. For example:

\begin{quote}

\textbf{Input}: $A[1...n]$  // Positive real number, distinct.

\textbf{Output}: $MAX A[i], 1<= i <= n$.

\end{quote}

\subsubsection{Algorithm 1}

Stupid way.

\begin{algorithm}[H]
\caption{Stupid Find Max Algorithm}\label{Stupid_Find_Max_Algorithm}
\begin{algorithmic}[1]
\Procedure{FindMax}{}
\For{$i = 1$ to $n$}
  \State $count = 0$
  \For{$j = 1$ to $n$}
    \If{$A[i] > A[j]$}
      \State $count = count + 1$
    \EndIf
  \EndFor
  \If{$count = n$}
    \State \textbf{return} {$A[i]$}
  \EndIf
\EndFor
\EndProcedure
\end{algorithmic}
\end{algorithm}

\textbf{Analysis}: Worst Case, $n^2$ comparison.

\subsubsection{Algorithm 2}

Sort \& Find.

\begin{algorithm}[H]
\caption{Sort \& Find Max Algorithm}\label{Sort_and_Find_Max_Algorithm}
\begin{algorithmic}[1]
\Procedure{FindMax}{}
\State {$\overline{A} = sort(A)$}
\State \textbf{return} {$\overline{A}[n]$}
\EndProcedure
\end{algorithmic}
\end{algorithm}

\textbf{Analysis}: Worst Case, sorting takes $c\ n\log n$ time.

\subsubsection{Algorithm 3}

Dynamically store the biggest one.

\begin{algorithm}[H]
\caption{Search \& Find Max Algorithm}\label{Search_and_Find_Max_Algorithm}
\begin{algorithmic}[1]
\Procedure{FindMax}{}
\State {$current = 1$}
\For{$i = 2$ to $n$}
  \If{$A[i] > A[current]$}
    \State{$current = i$}
  \EndIf
\EndFor
\State \textbf{return} {$A[current]$}
\EndProcedure
\end{algorithmic}
\end{algorithm}

\subsection{Can we do better?}

It depends on the operations allowed. For example the dropping the curtain and
find the first appearing one.
