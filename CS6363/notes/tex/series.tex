
\section{Series}
\subsection{Some Definition}

\begin{definition}
Harmonic Series:
\[\sum_{i=1}^n {\frac{1}{i}} = \Theta(\log(n))\]
\end{definition}

\begin{definition}
Geometric Series:
\[
\sum_{i=0}^{n-1}{x^i} = \frac{x^n - 1}{x-1} =
\begin{cases}
\Theta(x^n) & \text{if } \forall x>1, \\
\Theta(1) & \text{if } \forall x<1, \\
\Theta(n) & \text{if } \forall x=1.
\end{cases}
\]
\end{definition}

\begin{definition}

Arithmetic Series:

\[\sum_{i=1}^n {i} = \frac{n(n+1)}{2} = \Theta(n^2)\]

\end{definition}

\subsection{Some Theorem}

Suppose I want to know if $f(n) = o(g(n))$.

\begin{theorem}
If $\log(f(n)) = o(\log(g(n))$, then $f(n) = o(g(n))$.
\end{theorem}

Example: Let $f(n) = n^3$, $g(n) = 2^n$. Then $\log(f(n)) = \log(n^3) = 3\log(n)$, $\log(g(n)) = \log(2^n) = n$.
\[\text{i.e. } \log(f(n)) < \log(g(n) \Rightarrow f(n) < g(n)\]

\emph{Note that this theorem stands for `o', NOT TRUE for `$\mathcal{O}$'.}

Example: $\log(n^3) = \mathcal{O}(\log(n^2))$, but $n^3 \neq \mathcal{O}(n^2)$.

\section{Induction}
\subsection{When to use?}

Prove statement for all $n \in \mathbb{N}$, s.t. $n \geq n_0$.

\subsection{Definition}
Basically, induction has two parts:
\begin{enumerate}
\item {Base case(s) -- Sometimes there are more than one base cases.

Prove statement for some $n$. -- Often $n_0 = 0 \text{ or } 1$.}

\item {Induction Hypothesis

Assume statement hold true for all $m \leq n$.

Prove the hypothesis implies that it hold true for $n+1$.}
\end{enumerate}

Note that the process may be different from previous, which just hypothesize $n-1$ is true and prove for $n$.

\subsection{Example}
\subsubsection{Good Induction: Prove $\sum _{i=1}^n i = \frac{n(n+1)}{2}$}

\begin{proof}
We are required to prove $\forall n > 0 \text{, } \sum _{i=1}^n i = \frac{n(n+1)}{2}$.


\BaseCase $n=1$, $\sum _{i=1}^1 i = 1 = \frac{1\times (1+1)}{2}$. Hence the claim holds true for $n=1$.

\InductionStep Let $k > 1$ be an arbitrary natural number.

Let us assume the induction hypothesis: for every $k < n$, assume $\sum _{i=1}^k i = \frac{k(k+1)}{2}$. We will prove $\sum _{i=1}^{k+1} i = \frac{(k+1)(k+2)}{2}$

\begin{equation}
\sum _{i=1}^{k+1} i = \left(\sum _{i=1}^{k} i \right) + (n+1)
                    = \frac{k(k+1)}{2} + \frac{2(k+1)}{2}
                    = \frac{(k+1)(k+2)}{2}
\end{equation}

Thus establishes the claim for $k+1$.

\InductionConclusion By the principle of mathematical induction, the claim holds for all $n$.
\end{proof}

\subsubsection{Bad Induction: Prove all horses are the same color}

The process is omitted. The key point is that: if the base case is not true for induction hypothesis, the induction will not be solid.




